\documentclass[11pt, oneside]{article}
\usepackage{geometry} 
\geometry{letterpaper}
\usepackage{graphicx}	
\usepackage{amssymb, amsmath}
\usepackage{url}

\begin{document}

Let $X = X_{1} \times \ldots X_{N}$ with each
component having the same distribution, $\pi$,
such that the joint probability density function 
on $X$ is given by 
%
\begin{equation*}
\pi(x_{1}, \ldots, x_{N})
=
\prod_{n = 1}^{N} \pi(x_{n}).
\end{equation*}
%
Additionally consider the weighted summation 
function
%
\begin{equation*}
s_{w} : (x_{1}, \ldots, x_{N}) \mapsto \sum_{n = 1}^{N} w_{n} \, x_{n}.
\end{equation*}

If the component distribution is represented by
a Gaussian probability density function with locations
$\mu_{n}$ and scales $\sigma_{n}$ then the pushforward 
distribution of the joint distribution along $s_{w}$ 
is 
%
\begin{equation*}
\pi(s) = \mathcal{N} \left( s ; 
\sum_{n = 1}^{N} w_{n} \, \mu_{n}, 
\sqrt{ \sum_{n = 1}^{N} w^{2}_{n} \, \sigma^{2}_{n}} \right).
\end{equation*}
%
In the special case of identically distributed
components the pushforward distribution becomes
%
\begin{equation*}
\pi(s) = \mathcal{N} \left( s ; 
\sum_{n = 1}^{N} w_{n} \cdot \mu, 
\sqrt{ \sum_{n = 1}^{N} w^{2}_{n} } \cdot \sigma \right).
\end{equation*}


If the component distribution is represented by
a Cauchy probability density function with locations
$\mu_{n}$ and scales $\sigma_{n}$ then the pushforward 
distribution of the joint distribution along $s_{w}$ 
is 
%
\begin{equation*}
\pi(s) = \mathcal{C} \left( s ; 
\sum_{n = 1}^{N} w_{n} \, \mu_{n}, 
\sum_{n = 1}^{N} w_{n} \, \sigma_{n} \right).
\end{equation*}
%
In the special case of identically distributed
components the pushforward distribution becomes
%
\begin{equation*}
\pi(s) = \mathcal{N} \left( s ; 
\sum_{n = 1}^{N} w_{n} \cdot \mu, 
\sum_{n = 1}^{N} w_{n} \cdot \sigma \right).
\end{equation*}

Notice the difference -- in the Cauchy case the 
scales add linearly with the weights while in the
Gaussian case the scales add quadratically with
the weights.  This has an important difference 
when considering averages with $w_{n} = 1 / N$.
Here $\sum_{n = 1}^{N} w_{n} = 1$ and the
distribution of the average for the Cauchy is
the same as the distribution for any individual,
where as the distribution for the Gaussian average
concentrates around the common location. 


\end{document}  