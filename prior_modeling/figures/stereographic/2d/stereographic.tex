\documentclass{standalone}
\usepackage{graphicx}	
\usepackage{amssymb, amsmath, amsthm}
\usepackage{color}

\usepackage{tikz}
\usetikzlibrary{intersections, backgrounds, math}

\definecolor{light}{RGB}{220, 188, 188}
\definecolor{mid}{RGB}{185, 124, 124}
\definecolor{dark}{RGB}{143, 39, 39}
\definecolor{highlight}{RGB}{180, 31, 180}
\definecolor{gray10}{gray}{0.1}
\definecolor{gray20}{gray}{0.2}
\definecolor{gray30}{gray}{0.3}
\definecolor{gray40}{gray}{0.4}
\definecolor{gray60}{gray}{0.6}
\definecolor{gray70}{gray}{0.7}
\definecolor{gray80}{gray}{0.8}
\definecolor{gray90}{gray}{0.9}
\definecolor{gray60}{gray}{0.95}

\tikzmath{
  function x(\x, \y) {
    \xa = \x / (1 + \x * \x + \y * \y);
    return \xa;
    %return \xa * (-0.4 * \y + 2) / 2;
  };
  function y(\x, \y, \phi) {
    \sumsq = \x * \x + \y * \y;
    \denom = 1 / (1 + \sumsq);
    \ya = \y * \denom;
    \za = \sumsq * \denom;
    \yb = \ya * cos(\phi) - \za * sin(\phi);
    return \yb;
  };
  function z(\x, \y, \phi) {
    \sumsq = \x * \x + \y * \y;
    \denom = 1 / (1 + \sumsq);
    \ya = \y * \denom;
    \za = \sumsq * \denom;
    \zb = \za * cos(\phi) + \ya * sin(\phi);
    return \zb;
  };
  function xi(\x, \y) {
    return \x * (-0.4 * \y + 2) / 2;
  };
  function zeta(\y, \phi) {
    return \y * sin(\phi);
  };
  function mineta(\x, \phi) {
    \sphi = sin(\phi);
    \xsq = \x * \x;
    \spsi = (1 - \xsq) / (1 + \xsq);
    return \spsi * (1 - sqrt(1 - \sphi * \sphi * (1 + \xsq) )) / \sphi;
  };
  function maxeta(\x, \phi) {
    return (\x * \x - 1) / 6;
  };
}

\begin{document}

\begin{tikzpicture}[scale=0.3]
  \pgfmathsetmacro{\phi}{15}
  \pgfmathsetmacro{\lmin}{-1}
  \pgfmathsetmacro{\lmax}{1}
  \pgfmathsetmacro{\scale}{10}
  
\begin{scope}[shift={(0, 0)}]
  
    \draw[white] (-13, -6) rectangle (13, 12);
    
    % Linear density
    \fill[light]    ({\scale * xi(\lmin, \lmin)}, {\scale * zeta(\lmin, \phi)})
                 -- ({\scale * xi(\lmin, \lmax)}, {\scale * zeta(\lmax, \phi)})
                 -- ({\scale * xi(\lmax, \lmax)}, {\scale * zeta(\lmax, \phi)}) 
                 -- ({\scale * xi(\lmax, \lmin)}, {\scale * zeta(\lmin, \phi)})
                 -- cycle;
    
    % Horizontal lines
    \foreach \xi in {-0.8, -0.6, ..., 0.8} {
      \draw[mid]    ({\scale * xi(\xi, \lmin)}, {\scale * zeta(\lmin, \phi)}) 
                    -- ({\scale * xi(\xi, \lmax)}, {\scale * zeta(\lmax, \phi)});
    }
    
    % Vertical lines
    \foreach \eta in {-0.8, -0.6, ..., 0.8} {
      \draw[mid]    ({\scale * xi(\lmin, \eta)}, {\scale * zeta(\eta, \phi)}) 
                    -- ({\scale * xi(\lmax, \eta)}, {\scale * zeta(\eta, \phi)});
    }
  
    % Linear boundary
    \draw [gray80, line width=1.1, domain=0:355, samples=50, variable=\theta] 
      plot ({\scale * xi(0.8 * cos(\theta), 0.8 * sin(\theta)}, 
            {\scale * zeta(0.8 * sin(\theta), \phi)});
   
    \node[] at (0, -5) {$\mathbb{R}^{2}$};
   
  \end{scope}

  
  \begin{scope}[shift={(30, 0)}]
  
    \draw[white] (-13, -6) rectangle (13, 12);
    
    % Linear density
    \fill[light]    ({\scale * xi(\lmin, \lmin)}, {\scale * zeta(\lmin, \phi)})
                 -- ({\scale * xi(\lmin, \lmax)}, {\scale * zeta(\lmax, \phi)})
                 -- ({\scale * xi(\lmax, \lmax)}, {\scale * zeta(\lmax, \phi)}) 
                 -- ({\scale * xi(\lmax, \lmin)}, {\scale * zeta(\lmin, \phi)})
                 -- cycle;
    
    % Horizontal lines
    \foreach \xi in {-0.8, -0.6, ..., 0.8} {
      \draw[mid]    ({\scale * xi(\xi, \lmin)}, {\scale * zeta(\lmin, \phi)}) 
                    -- ({\scale * xi(\xi, \lmax)}, {\scale * zeta(\lmax, \phi)});
    }
    
    % Vertical lines
    \foreach \eta in {-0.8, -0.6, ..., 0.8} {
      \draw[mid]    ({\scale * xi(\lmin, \eta)}, {\scale * zeta(\eta, \phi)}) 
                    -- ({\scale * xi(\lmax, \eta)}, {\scale * zeta(\eta, \phi)});
    }
  
    % Linear boundary
    \draw [gray80, line width=1.1, domain=0:355, samples=50, variable=\theta] 
      plot ({\scale * xi(0.8 * cos(\theta), 0.8 * sin(\theta)}, 
            {\scale * zeta(0.8 * sin(\theta), \phi)});
    
    % Sphere outline
    \fill[white] (0, {\scale * 0.5 * cos(\phi)}) circle ({\scale * 0.5});
    
    % Spherical density
    \foreach \theta in {180, 179.25, ..., 0} {
      \pgfmathsetmacro{\prop}{100 * exp(-\theta / 135.0)}  
      \colorlet{custom}{mid!\prop!white}
      \fill[color=custom] 
        (0, {cos(\phi) * \scale * 1. * 0.5 * (1 + cos(\theta))}) 
          circle ({\scale * 1.001 * 0.5 * sin(\theta)} and {sin(\phi) * \scale * 1.001 * 0.5 * sin(\theta)});
    }
    
    % Background line aggregate
    \begin{scope} {
      \clip ({-\scale * 0.25}, {\scale * 0.905}) rectangle ({\scale * 0.25}, {\scale * 0.5 * (1 + cos(\phi))});
      \fill[mid] (0, {\scale * 0.5 * cos(\phi)}) circle ({\scale * 0.5});
    }
    \end{scope}
    
    % Background stereographic vertical lines
    \foreach \xi in {-8.0, -7.8, ..., 8.0} {
      \draw [mid, domain=-8:{maxeta(\xi, \phi)},, samples=60, variable=\eta] 
        plot ({\scale * x(\xi, \eta)}, {\scale * z(\xi, \eta, \phi)});
    }
    
    % Background stereographic horizontal lines
    \foreach \eta in {-1.2, -1.4, ..., -8} {
      \draw [mid, domain=-8:8, samples=100, variable=\xi] 
        plot ({\scale * x(\xi, \eta)}, {\scale * z(\xi, \eta, \phi});
    }
  
    \foreach \eta in {-0.2, -0.4, ..., -1}  {
      \draw [mid, domain=-10:10, samples=300, variable=\xi] 
        plot ({\scale * x(\xi, \eta)}, {\scale * z(\xi, \eta, \phi});
    }
    
    \pgfmathsetmacro{\eta}{0.0}
    \draw [mid, domain=-1:-10, samples=100, variable=\xi] 
      plot ({\scale * x(\xi, \eta)}, {\scale * z(\xi, \eta, \phi});
    \draw [mid, domain=1:10, samples=100, variable=\xi] 
      plot ({\scale * x(\xi, \eta)}, {\scale * z(\xi, \eta, \phi});
  
    \pgfmathsetmacro{\eta}{0.2}
    \draw [mid, domain=-1.25:-10, samples=100, variable=\xi] 
      plot ({\scale * x(\xi, \eta)}, {\scale * z(\xi, \eta, \phi});
    \draw [mid, domain=1.25:10, samples=100, variable=\xi] 
      plot ({\scale * x(\xi, \eta)}, {\scale * z(\xi, \eta, \phi});
  
    \pgfmathsetmacro{\eta}{0.4}
    \draw [mid, domain=-1.5:-10, samples=100, variable=\xi] 
      plot ({\scale * x(\xi, \eta)}, {\scale * z(\xi, \eta, \phi});
    \draw [mid, domain=1.5:10, samples=100, variable=\xi] 
      plot ({\scale * x(\xi, \eta)}, {\scale * z(\xi, \eta, \phi});
      
    % Stereographic boundary
    \draw [gray80, line width=1.1, domain=0:-180, samples=30, variable=\theta] 
      plot ({\scale * x(0.8 * cos(\theta), 0.8 * sin(\theta)}, 
            {\scale * z(0.8 * cos(\theta), 0.8 * sin(\theta), \phi)});
    
    % Point at infinity
    \node[] at (0, {\scale * 1.05}) {$\infty$};
    \fill[color=dark] (0, {\scale * cos(\phi)}) circle (5pt);
   
    \node[] at (0, -5) {Stereographic Projection of $\mathbb{R}^{2}$};
   
  \end{scope}
 
\end{tikzpicture}

\end{document}  