\documentclass{standalone}
\usepackage{graphicx}	
\usepackage{amssymb, amsmath}
\usepackage{color}

\usepackage{tikz}
\usetikzlibrary{intersections, backgrounds, math}
\usepackage{pgfmath}

\definecolor{light}{RGB}{220, 188, 188}
\definecolor{mid}{RGB}{185, 124, 124}
\definecolor{dark}{RGB}{143, 39, 39}
\definecolor{highlight}{RGB}{180, 31, 180}
\definecolor{gray10}{gray}{0.1}
\definecolor{gray20}{gray}{0.2}
\definecolor{gray30}{gray}{0.3}
\definecolor{gray40}{gray}{0.4}
\definecolor{gray60}{gray}{0.6}
\definecolor{gray70}{gray}{0.7}
\definecolor{gray80}{gray}{0.8}
\definecolor{gray90}{gray}{0.9}
\definecolor{gray95}{gray}{0.95}

\tikzmath{
  function gamma(\x) {
    return exp( -17.36233 + (14.7212 - 1) * ln(\x) - 1.61825 * \x);
  };
  function inv_gamma(\x) {
    return exp( 45.12796 + -(14.7212 + 1) * ln(\x) - 112.873 / \x);
  };
}

\begin{document}

\begin{tikzpicture}[scale=0.25, thick]

  \begin{scope}[shift={(0, 0)}]
    \draw[white] (-12, -4.5) rectangle (12, 18.5);
    
    \node at (0, 17) { Gamma Containment };
  
    \node[align=center] at (-5.5, 13) { Lower\\Extremity\\Threshold };
    \node[align=center] at (+5.5, 13) { Upper\\Extremity\\Threshold };
  
    \fill[color=gray80, opacity=0.33] (-5.5, 0) rectangle (5.5, 10.5);
    
    \draw[domain={-9.99:10}, smooth, samples=150, line width=1, variable=\x, color=dark] 
      plot ({\x},{50 * gamma(\x + 10))});

    \begin{scope}
      \clip (-10, 0) rectangle (-5.5, 10);
      \fill[domain={-9.99:10}, smooth, samples=150, line width=1, variable=\x, color=dark] 
        plot ({\x},{50 * gamma(\x + 10))});
    \end{scope}
    
    \begin{scope}
      \clip (5.5, 0) rectangle (10, 10);
      \fill[domain={-9.99:10}, smooth, samples=150, line width=1, variable=\x, color=dark] 
        plot ({\x},{50 * gamma(\x + 10))});
    \end{scope}
    
    \node[dark] at (-7.5, 1.5) { $1\%$ };
    \node[dark] at (7.5, 1.5) { $1\%$ };
    
    \draw[-, color=gray70, line width=1] (-5.5, 0) -- (-5.5, 10.5);
    \draw[->, >=stealth, line width=1, color=gray70] (-5.5, 5) -- +(1.5, 0);
  
    \draw[-, color=gray70, line width=1] (5.5, 0) -- (5.5, 10.5);
    \draw[->, >=stealth, line width=1, color=gray70] (5.5, 5) -- +(-1.5, 0);
  
    \draw[->, >=stealth, line width=1] (-10, -0.05) -- +(0, 11);
    \draw[->, >=stealth, line width=1] (-10.05, 0) -- +(20.5, 0);
    
    \node[] at (-10, -1) { $0$ };
    \node[] at (+10, -1) { $+\infty$ };
    
    \node[] at (0, -3) { $\theta$ };
    
  \end{scope}
  
  \begin{scope}[shift={(26, 0)}]
    \draw[white] (-12, -4.5) rectangle (12, 18.5);
    
    \node at (0, 17) { Inverse Gamma Containment };
  
    \node[align=center] at (-5.5, 13) { Lower\\Extremity\\Threshold };
    \node[align=center] at (+5.5, 13) { Upper\\Extremity\\Threshold };
  
    \fill[color=gray80, opacity=0.33] (-5.5, 0) rectangle (5.5, 10.5);
    
    \draw[domain={-9.99:10}, smooth, samples=150, line width=1, variable=\x, color=dark] 
      plot ({\x},{50 * inv_gamma(\x + 10))});

    \begin{scope}
      \clip (-10, 0) rectangle (-5.5, 10);
      \fill[domain={-9.99:10}, smooth, samples=150, line width=1, variable=\x, color=dark] 
        plot ({\x},{50 * inv_gamma(\x + 10))});
    \end{scope}
    
    \begin{scope}
      \clip (5.5, 0) rectangle (10, 10);
      \fill[domain={-9.99:10}, smooth, samples=150, line width=1, variable=\x, color=dark] 
        plot ({\x},{50 * inv_gamma(\x + 10))});
    \end{scope}
    
    \node[dark] at (-7.5, 1.5) { $1\%$ };
    \node[dark] at (7.5, 1.5) { $1\%$ };
    
    \draw[-, color=gray70, line width=1] (-5.5, 0) -- (-5.5, 10.5);
    \draw[->, >=stealth, line width=1, color=gray70] (-5.5, 5) -- +(1.5, 0);
  
    \draw[-, color=gray70, line width=1] (5.5, 0) -- (5.5, 10.5);
    \draw[->, >=stealth, line width=1, color=gray70] (5.5, 5) -- +(-1.5, 0);
  
    \draw[->, >=stealth, line width=1] (-10, -0.05) -- +(0, 11);
    \draw[->, >=stealth, line width=1] (-10.05, 0) -- +(20.5, 0);
    
    \node[] at (-10, -1) { $0$ };
    \node[] at (+10, -1) { $+\infty$ };
    
    \node[] at (0, -3) { $\theta$ };
    
  \end{scope}
  
\end{tikzpicture}

\end{document}  